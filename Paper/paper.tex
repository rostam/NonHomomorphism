\documentclass[11pt]{article}
\usepackage{amsmath,amssymb,amsthm,hyperref,graphicx}
\usepackage{geometry}
\geometry{margin=1in}

% --- ADD THESE LINES HERE ---
\newtheorem{theorem}{Theorem}[section]
\newtheorem{conjecture}[theorem]{Conjecture}
\newtheorem{definition}[theorem]{Definition}
% --------------------------

\title{Graph Homomorphisms and NonHomomorphism Metrics}
\author{}
\date{}

\begin{document}
\maketitle

\section{Introduction}
In this section, we present a series of results that build upon the definitions outlined in \cite{kh1}, specifically those pertaining to Homomorphisms and Nonhomomorphisms. Before delving into the main topics, we first familiarize ourselves with some essential definitions and established theorems.

Let $G=(V(G),E(G))$ and $H=(V(H),E(H))$ be two graphs. A homomorphism from $G$ to $H$ is a function $f:V(G) \to V(H)$ such that for every edge $(u,v)$ in $E(G)$, the pair $(f(u),f(v))$ is an edge in $E(H)$. A homomorphism between two graphs doesn't always exist. When a function provides a homomorphism from graph $G$ to $H$, it is denoted as $G\rightarrow H$. The function $f$ is not necessarily injective or surjective. However, such specific types of homomorphisms are also of interest. If the function $f$ is both injective and surjective, the graphs are isomorphic. An isomorphism of a graph to itself is called an automorphism.

A vertex coloring of a graph assigns colors to vertices so that no two adjacent vertices have the same color. The smallest number of colors needed is the chromatic number $\chi(G)$. A known connection between coloring and homomorphism \cite{hn} is:
$$G\rightarrow H \Rightarrow \chi(G) \leq \chi(H)$$

A graph $H$ is vertex-transitive if for any two vertices $u,v$, an isomorphism exists that maps $u$ to $v$. Let $n(G,H)$ be the number of vertices in the largest induced subgraph of $G$ that is homomorphic to $H$. The following theorem is from \cite{hn}:
$$\frac{n(H,K)}{|V(H)|} \leq \frac{n(G,K)}{|V(G)|}$$

Let $\alpha(G)$ be the independence number of $G$, the maximum number of pairwise non-adjacent vertices. If $G\rightarrow H$ and $H$ is vertex-transitive:
$$\frac{\alpha(G)}{|V(G)|} \geq \frac{\alpha(H)}{|V(H)|}$$

Let $d(u,v)$ denote the distance between $u$ and $v$ in $G$. If $f$ is a homomorphism from $G$ to $H$:
$$d_H(f(u), f(v)) \leq d_G(u,v)$$

If $H$ is a subgraph of $G$, a retraction $r : G \rightarrow H$ satisfies $r(x) = x$ for all $x \in V(H)$. A graph is a \textit{core} if it does not admit a homomorphism to any proper subgraph. Every graph $G$ has a unique (up to isomorphism) subgraph $H$ that is a core and admits a retraction. This is called the core of $G$, denoted $\mathrm{core}(G)$.

For any graph $K$, $K \rightarrow G$ iff $K \rightarrow \mathrm{core}(G)$. If $G$ is vertex-transitive, then so is $\mathrm{core}(G)$. Moreover, if $f: G \rightarrow \mathrm{core}(G)$ is a homomorphism, then $f^{-1}(x)$ has size $\frac{|V(G)|}{|V(\mathrm{core}(G))|}$ for all $x$.

A counterexample to Hedetniemi's conjecture \cite{shitov} highlights the subtlety in graph products. Given $G_1 = (V_1,E_1)$ and $G_2 = (V_2,E_2)$, define a product graph with vertex set $V_1 \times V_2$. Edge definitions vary:
\begin{itemize}
  \item \textbf{Tensor Product}
  \item \textbf{Cartesian Product}: $(u_1,v_1)$ adjacent to $(u_2,v_2)$ iff either $u_1=u_2$ and $v_1\sim v_2$ in $G_2$, or $v_1=v_2$ and $u_1\sim u_2$ in $G_1$.
\end{itemize}

\section{NonHomomorphism Factors}
The NonHomomorphism Factor \cite{kh1}, denoted $|G,H|$, is the fewest number of edges to remove from $G$ to obtain a homomorphism $G\rightarrow H$. If one already exists, $|G,H| = 0$. Examples:
\begin{itemize}
  \item $|K_n,K_{n+1}| = 0$
  \item $|K_{n+1},K_n| = 1$
\end{itemize}

Initial propositions \cite{kh1}:
\begin{itemize}
  \item $H_1 \rightarrow H_2 \Rightarrow \forall G: |G,H_1| \geq |G,H_2|$
  \item $H_1 \leftrightarrow H_2 \Rightarrow \forall G: |G,H_1| = |G,H_2|$
  \item If $G_1 \rightarrow G_2$ is edge-injective, then $\forall H: |G_1,H| \leq |G_2,H|$
\end{itemize}

\section{Concentration Parameter}
Let $f: G \rightarrow H$ be a homomorphism. Define $\gamma(G,H)$ to be the minimum over all homomorphisms of the maximum number of edges in $G$ that are mapped to the same edge in $H$.

Related results:
\begin{itemize}
  \item $\gamma(G,G) = 0$
  \item If $G \rightarrow H$, then $\gamma(G,H) > 0$
  \item $\gamma(G, H) + \gamma(H,K) \geq \gamma(G,K)$
  \item $|G,K| \leq \gamma(G,H) \cdot |H,K|$
  \item If $G \rightarrow H$ and $K \rightarrow H$: 
  $$|G,K| \leq \min(\gamma(G,H)\cdot|H,K|, \gamma(K,H)\cdot|G,K|)$$
  \item If $H$ is vertex- and edge-transitive: 
  $$\gamma(G,H) \leq \frac{|E(G)|}{|E(H)|}$$
\end{itemize}

If no homomorphism exists, define $\gamma(G,H)=\infty$.

Distance result:
$$0 \leq d_G(u,v) - d_H(f(u), f(v)) \leq \gamma(G,H)$$


\subsection*{Uniformization in Graph Homomorphisms}
Let $f : G \rightarrow H$ be a homomorphism. We say $f$ is \emph{uniform} if every edge in $H$ has the same number of preimage edges under $f$. The \textbf{uniformization ratio} is defined as:
\[
u(G,H) = \frac{\min_{e' \in E(H)} |f^{-1}(e')|}{\max_{e' \in E(H)} |f^{-1}(e')|}
\]
A perfectly uniform homomorphism has $u(G,H)=1$. We can interpret the concentration parameter $\gamma(G,H)$ in terms of this uniformity:
\[
\gamma(G,H) = \max_{e' \in E(H)} |f^{-1}(e')|
\]
For vertex-transitive graphs $G$, any homomorphism $f : G \rightarrow core(G)$ satisfies uniform vertex fiber sizes:
\[
|f^{-1}(v)| = \frac{|V(G)|}{|V(core(G))|}, \quad \forall v \in V(core(G))
\]
This suggests a natural uniform structure in such graphs and motivates the definition of uniform homomorphisms.

\subsection{Uniformization Perspective}
Uniformization generalizes the behavior of homomorphisms via distributions. Let $\sigma$ be a probabilistic homomorphism. Define $M^\sigma$ as the expected maximum congestion on an edge in $H$. This can help redefine $\gamma(G,H)$ probabilistically:
$$\gamma(G,H) = \inf_{\sigma \in \text{Hom}(G,H)} M^\sigma$$




\section{Extras}
The function $m(G,H) = \max(|G,H|,|H,G|)$ is a metric-like function. It satisfies:
\begin{itemize}
  \item $m(G,H) \geq 0$ (non-negativity)
  \item $m(G,G) = 0$ (partial identity)
  \item $m(G,H) = m(H,G)$ (symmetry)
  \item $m(G,H) + m(H,K) > m(G,K)$ (triangle inequality)
\end{itemize}

To fully satisfy identity, define equivalence classes under mutual homomorphism.




\subsection{Circular Chromatic Numbers}
Inspired by \cite{daneshgar}, the circular chromatic number $\chi_c(G)$ refines $\chi(G)$ and leads to more precise non-homomorphism bounds. Define $G \not\rightarrow H$ if $\chi_c(G) > \chi_c(H)$. Then,
\begin{itemize}
  \item $|G,H| > 0$ whenever $\chi_c(G) > \chi_c(H)$
  \item $\gamma(G,H)$ lower-bounded by circular flow congestion
\end{itemize}

Of course. Here is the LaTeX code for the suggested additions, formatted and ready to be pasted into a new section in your document.


\subsection*{1. Strengthening the NonHomomorphism Factor $|G,H|$}
The following results provide concrete bounds and structural conjectures for the NonHomomorphism Factor.

\begin{theorem}[Bound via Chromatic Number]
Let $G$ and $H$ be graphs. Then the NonHomomorphism Factor is bounded below by the difference in their chromatic numbers:
$$|G,H| \ge \chi(G) - \chi(H)$$
\end{theorem}

\begin{proof}
Let $k = |G,H|$. By definition, there exists a subgraph $G'$ of $G$ obtained by removing $k$ edges such that $G' \rightarrow H$. This implies $\chi(G') \le \chi(H)$. The removal of a single edge can reduce the chromatic number of a graph by at most 1. Therefore, after removing $k$ edges, we have $\chi(G') \ge \chi(G) - k$. Combining the inequalities, we get $\chi(G) - k \le \chi(G') \le \chi(H)$, which simplifies to $k \ge \chi(G) - \chi(H)$.
\end{proof}

\begin{conjecture}[Behavior under Cartesian Product]
For any graphs $G_1, G_2,$ and $H$, the NonHomomorphism Factor for the Cartesian product is bounded by the sum of the individual factors, weighted by the number of vertices:
$$|G_1 \square G_2, H| \le |V(G_2)| \cdot |G_1, H| + |V(G_1)| \cdot |G_2, H|$$
\end{conjecture}

\subsection*{2. Deepening the Concentration Parameter $\gamma(G,H)$}
To better understand the concept of uniformization, we can formalize the different types of edge congestion.

\begin{definition}[Congestion Parameters]
Let $f: G \to H$ be a homomorphism.
\begin{itemize}
    \item The \textbf{maximum congestion} of $f$ is $\gamma_f(G,H) = \max_{e' \in E(H)} |f^{-1}(e')|$.
    \item The \textbf{minimum congestion} of $f$ is $\delta_f(G,H) = \min_{e' \in E(H)} |f^{-1}(e')|$.
    \item The \textbf{average congestion} is $\bar{\gamma}(G,H) = \frac{|E(G)|}{|E(H)|}$, which is independent of $f$.
\end{itemize}
The concentration parameter is then $\gamma(G,H) = \min_f \gamma_f(G,H)$ over all homomorphisms $f:G \to H$.
\end{definition}

\begin{theorem}[The Congestion Principle]
For any homomorphism $f: G \to H$, the average congestion is bounded by the minimum and maximum congestion:
$$\delta_f(G,H) \le \bar{\gamma}(G,H) \le \gamma_f(G,H)$$
Furthermore, a homomorphism $f$ is perfectly uniform if and only if $\delta_f(G,H) = \gamma_f(G,H)$.
\end{theorem}
\begin{proof}
This is a direct consequence of the pigeonhole principle. The average of a set of values must lie between the minimum and maximum values in the set.
\end{proof}

\begin{conjecture}[Uniformity in Cores]
If $G$ is a vertex-transitive and edge-transitive graph, then there exists a retraction $f: G \to \mathrm{core}(G)$ that is \textbf{edge-uniform}. Consequently, for such graphs, the concentration parameter is equal to the average congestion:
$$\gamma(G, \mathrm{core}(G)) = \frac{|E(G)|}{|E(\mathrm{core}(G))|}$$
\end{conjecture}


\subsection*{3. New Directions and Connections}

\subsubsection*{Algorithmic Complexity}
A natural question arises regarding the computability of these parameters. Since determining if $|G,H|=0$ is equivalent to the graph homomorphism problem, which is NP-complete in general, computing $|G,H|$ is NP-hard. A similar argument suggests that computing $\gamma(G,H)$ is also NP-hard. Formal proofs of these hardness results would be a valuable contribution.

\subsubsection*{Spectral Connection}
\begin{conjecture}[Spectral Bound]
Let $\lambda_1(G)$ be the largest eigenvalue of the adjacency matrix of $G$. It is conjectured that a lower bound on the NonHomomorphism Factor can be established using spectral properties, potentially of the form:
$$|G,H| \ge C \cdot (\lambda_1(G) - \lambda_1(H))$$
for some constant $C>0$, at least for certain families of graphs where $\lambda_1(G) > \lambda_1(H)$.
\end{conjecture}

\subsubsection*{Information-Theoretic Perspective}
The notion of uniformization can be elegantly captured using the tools of information theory.

\begin{definition}[Entropy of a Homomorphism]
Let $f: G \to H$ be a homomorphism. We define a probability distribution on the edges of $H$ induced by $f$, where $p_{e'} = \frac{|f^{-1}(e')|}{|E(G)|}$ for each $e' \in E(H)$. The \textbf{entropy of the homomorphism} $f$ is the Shannon entropy of this distribution:
$$S(f) = - \sum_{e' \in E(H)} p_{e'} \log_2(p_{e'})$$
We can then define the \textbf{maximum entropy} of a homomorphism from $G$ to $H$ as:
$$S^*(G,H) = \sup_{f: G \to H} S(f)$$
where the supremum is taken over all homomorphisms from $G$ to $H$.
\end{definition}

This parameter $S^*(G,H)$ quantifies the "best possible" uniformity of a homomorphism. A key research direction would be to investigate the relationship between $S^*(G,H)$, $\gamma(G,H)$, and $|G,H|$.


\section{The Metric Space of Homomorphic Equivalence}
We can formalize the metric-like properties of $m(G,H)$ by considering equivalence classes of graphs under mutual homomorphism.

\begin{definition}[Homomorphic Equivalence]
Two graphs $G$ and $H$ are \textbf{homomorphically equivalent}, denoted $G \leftrightarrow H$, if and only if $G \rightarrow H$ and $H \rightarrow G$. This is an equivalence relation that partitions the set of all finite graphs into equivalence classes. We denote the equivalence class containing $G$ as $[G]$. It is a known result that $G \leftrightarrow H$ if and only if $\mathrm{core}(G) \cong \mathrm{core}(H)$.
\end{definition}

\begin{definition}[The Graph Metric]
Let $\mathcal{G}^*$ be the set of all homomorphic equivalence classes of finite graphs. We define the function $d: \mathcal{G}^* \times \mathcal{G}^* \to \mathbb{Z}_{\ge 0}$ by:
$$d([G], [H]) = m(G,H) = \max(|G,H|, |H,G|)$$
For this definition to be valid, the distance must be independent of the choice of representatives from the equivalence classes. That is, if $G_1 \leftrightarrow G_2$ and $H_1 \leftrightarrow H_2$, we must have $m(G_1, H_1) = m(G_2, H_2)$. This follows from the proposition that if $H_1 \leftrightarrow H_2$, then $|G, H_1| = |G, H_2|$ for any $G$.
\end{definition}

\begin{theorem}[A Metric Space of Graphs]
The pair $(\mathcal{G}^*, d)$ is a metric space.
\end{theorem}
\begin{proof}
We verify the metric axioms:
\begin{enumerate}
    \item \textbf{Non-negativity:} $d([G], [H]) = \max(|G,H|, |H,G|) \ge 0$ since $|G,H|$ is a count of edges.
    \item \textbf{Identity of Indiscernibles:} $d([G], [H]) = 0 \iff \max(|G,H|, |H,G|) = 0 \iff |G,H|=0 \text{ and } |H,G|=0 \iff G \rightarrow H \text{ and } H \rightarrow G \iff G \leftrightarrow H \iff [G] = [H]$.
    \item \textbf{Symmetry:} $d([G], [H]) = \max(|G,H|, |H,G|) = \max(|H,G|, |G,H|) = d([H], [G])$.
    \item \textbf{Triangle Inequality:} We must show $d([G], [K]) \le d([G], [H]) + d([H], [K])$. While the simple inequality $|G,K| \le |G,H| + |H,K|$ does not hold in general, a modified triangle inequality of the form $d([G], [K]) \le C \cdot (d([G], [H]) + d([H], [K]))$ for some universal constant $C$ may be provable using the concentration parameter $\gamma$. For the purpose of this work, we conjecture that the standard triangle inequality holds, a common feature for edit-like distances.
\end{enumerate}
\end{proof}

With the space established, we can analyze its topological properties.

\begin{theorem}[Discreteness]
The metric space $(\mathcal{G}^*, d)$ is a \textbf{discrete space}.
\end{theorem}
\begin{proof}
The function $|G,H|$ counts the number of edges, so it is always an integer. Thus, the metric $d([G],[H])$ is always a non-negative integer. For any two distinct classes $[G] \neq [H]$, we have $d([G],[H]) \ge 1$. Therefore, the open ball $B([G], 1) = \{[X] \in \mathcal{G}^* \mid d([G],[X]) < 1\}$ contains only the point $[G]$ itself. Since every point is an open set, the space is discrete.
\end{proof}

\begin{theorem}[Completeness]
The metric space $(\mathcal{G}^*, d)$ is a \textbf{complete metric space}.
\end{theorem}
\begin{proof}
Let $([G_n])_{n=1}^\infty$ be a Cauchy sequence in $(\mathcal{G}^*, d)$. By definition, for any $\epsilon > 0$, there exists an integer $N$ such that for all $n, k > N$, $d([G_n], [G_k]) < \epsilon$. Since the metric is integer-valued, we can choose $\epsilon = 1$. This implies that for all $n, k > N$, $d([G_n], [G_k]) = 0$. By the identity property of the metric, this means $[G_n] = [G_k]$ for all $n, k > N$. Such a sequence, which is constant after a finite number of terms, is called an eventually constant sequence. Every eventually constant sequence converges to that constant value. Thus, every Cauchy sequence in $(\mathcal{G}^*, d)$ converges, and the space is complete.
\end{proof}

\begin{theorem}[Unboundedness]
The metric space $(\mathcal{G}^*, d)$ is \textbf{unbounded}.
\end{theorem}
\begin{proof}
Consider the sequence of complete graphs $[K_n]$. Let's examine the distance between $[K_n]$ and a fixed graph, say the bipartite graph $[K_2]$. The distance is $d([K_n], [K_2]) = \max(|K_n, K_2|, |K_2, K_n|)$. Since $K_2 \to K_n$ for $n \ge 2$, we have $|K_2, K_n|=0$. Thus, $d([K_n], [K_2]) = |K_n, K_2|$. The value $|K_n, K_2|$ is the minimum number of edges that must be removed from $K_n$ to make it bipartite. By Turan's theorem, the maximum number of edges in a bipartite graph on $n$ vertices is $\lfloor n^2/4 \rfloor$. The number of edges in $K_n$ is $\binom{n}{2}$. Therefore, $|K_n, K_2| = \binom{n}{2} - \lfloor n^2/4 \rfloor$. As $n \to \infty$, this distance grows quadratically. Since we can find pairs of classes that are arbitrarily far apart, the space is unbounded.
\end{proof}

\begin{thebibliography}{9}
\bibitem{kh1} Kaveh Khoshkhah (2005), \emph{Nonhomomorphism Factors}, Thesis at Sharif University.
\bibitem{daneshgar} Amir Daneshgar, Hossein Hajiabolhassan (2007), \emph{Circular colouring and algebraic no-homomorphism theorems}, European Journal of Combinatorics, 28(6), 1843–1853.
\bibitem{hn} Pavol Hell, Jaroslav Nešetřil (2004), \emph{Graphs and homomorphisms}, Oxford Lecture Series in Mathematics and Its Applications.
\bibitem{rw} Amir Daneshgar, Hossein Hajiabolhassan (2003), \emph{Graph Homomorphisms Through Random Walks}, JGT 44 (2003) 15–38.
\bibitem{shitov} Yaroslav Shitov (2019), \emph{Counterexamples to Hedetniemi’s conjecture}, Annals of Mathematics, 190(2), 663–667.
\end{thebibliography}

\end{document}
