\documentclass[11pt]{article}
\usepackage{amsmath,amssymb,amsthm,hyperref,graphicx}
\usepackage{geometry}
\geometry{margin=1in}
\title{Graph Homomorphisms and NonHomomorphism Metrics}
\author{}
\date{}

\begin{document}
\maketitle

\section{Introduction}
In this section, we present a series of results that build upon the definitions outlined in \cite{kh1}, specifically those pertaining to Homomorphisms and Nonhomomorphisms. Before delving into the main topics, we first familiarize ourselves with some essential definitions and established theorems.

Let $G=(V(G),E(G))$ and $H=(V(H),E(H))$ be two graphs. A homomorphism from $G$ to $H$ is a function $f:V(G) \to V(H)$ such that for every edge $(u,v)$ in $E(G)$, the pair $(f(u),f(v))$ is an edge in $E(H)$. A homomorphism between two graphs doesn't always exist. When a function provides a homomorphism from graph $G$ to $H$, it is denoted as $G\rightarrow H$. The function $f$ is not necessarily injective or surjective. However, such specific types of homomorphisms are also of interest. If the function $f$ is both injective and surjective, the graphs are isomorphic. An isomorphism of a graph to itself is called an automorphism.

A vertex coloring of a graph assigns colors to vertices so that no two adjacent vertices have the same color. The smallest number of colors needed is the chromatic number $\chi(G)$. A known connection between coloring and homomorphism \cite{hn} is:
$$G\rightarrow H \Rightarrow \chi(G) \leq \chi(H)$$

A graph $H$ is vertex-transitive if for any two vertices $u,v$, an isomorphism exists that maps $u$ to $v$. Let $n(G,H)$ be the number of vertices in the largest induced subgraph of $G$ that is homomorphic to $H$. The following theorem is from \cite{hn}:
$$\frac{n(H,K)}{|V(H)|} \leq \frac{n(G,K)}{|V(G)|}$$

Let $\alpha(G)$ be the independence number of $G$, the maximum number of pairwise non-adjacent vertices. If $G\rightarrow H$ and $H$ is vertex-transitive:
$$\frac{\alpha(G)}{|V(G)|} \geq \frac{\alpha(H)}{|V(H)|}$$

Let $d(u,v)$ denote the distance between $u$ and $v$ in $G$. If $f$ is a homomorphism from $G$ to $H$:
$$d_H(f(u), f(v)) \leq d_G(u,v)$$

If $H$ is a subgraph of $G$, a retraction $r : G \rightarrow H$ satisfies $r(x) = x$ for all $x \in V(H)$. A graph is a \textit{core} if it does not admit a homomorphism to any proper subgraph. Every graph $G$ has a unique (up to isomorphism) subgraph $H$ that is a core and admits a retraction. This is called the core of $G$, denoted $\mathrm{core}(G)$.

For any graph $K$, $K \rightarrow G$ iff $K \rightarrow \mathrm{core}(G)$. If $G$ is vertex-transitive, then so is $\mathrm{core}(G)$. Moreover, if $f: G \rightarrow \mathrm{core}(G)$ is a homomorphism, then $f^{-1}(x)$ has size $\frac{|V(G)|}{|V(\mathrm{core}(G))|}$ for all $x$.

A counterexample to Hedetniemi's conjecture \cite{shitov} highlights the subtlety in graph products. Given $G_1 = (V_1,E_1)$ and $G_2 = (V_2,E_2)$, define a product graph with vertex set $V_1 \times V_2$. Edge definitions vary:
\begin{itemize}
  \item \textbf{Tensor Product}
  \item \textbf{Cartesian Product}: $(u_1,v_1)$ adjacent to $(u_2,v_2)$ iff either $u_1=u_2$ and $v_1\sim v_2$ in $G_2$, or $v_1=v_2$ and $u_1\sim u_2$ in $G_1$.
\end{itemize}

\section{NonHomomorphism Factors}
The NonHomomorphism Factor \cite{kh1}, denoted $|G,H|$, is the fewest number of edges to remove from $G$ to obtain a homomorphism $G\rightarrow H$. If one already exists, $|G,H| = 0$. Examples:
\begin{itemize}
  \item $|K_n,K_{n+1}| = 0$
  \item $|K_{n+1},K_n| = 1$
\end{itemize}

Initial propositions \cite{kh1}:
\begin{itemize}
  \item $H_1 \rightarrow H_2 \Rightarrow \forall G: |G,H_1| \geq |G,H_2|$
  \item $H_1 \leftrightarrow H_2 \Rightarrow \forall G: |G,H_1| = |G,H_2|$
  \item If $G_1 \rightarrow G_2$ is edge-injective, then $\forall H: |G_1,H| \leq |G_2,H|$
\end{itemize}

\section{Concentration Parameter}
Let $f: G \rightarrow H$ be a homomorphism. Define $\gamma(G,H)$ to be the minimum over all homomorphisms of the maximum number of edges in $G$ that are mapped to the same edge in $H$.

Related results:
\begin{itemize}
  \item $\gamma(G,G) = 0$
  \item If $G \rightarrow H$, then $\gamma(G,H) > 0$
  \item $\gamma(G, H) + \gamma(H,K) \geq \gamma(G,K)$
  \item $|G,K| \leq \gamma(G,H) \cdot |H,K|$
  \item If $G \rightarrow H$ and $K \rightarrow H$: 
  $$|G,K| \leq \min(\gamma(G,H)\cdot|H,K|, \gamma(K,H)\cdot|G,K|)$$
  \item If $H$ is vertex- and edge-transitive: 
  $$\gamma(G,H) \leq \frac{|E(G)|}{|E(H)|}$$
\end{itemize}

If no homomorphism exists, define $\gamma(G,H)=\infty$.

Distance result:
$$0 \leq d_G(u,v) - d_H(f(u), f(v)) \leq \gamma(G,H)$$


\subsection*{Uniformization in Graph Homomorphisms}
Let $f : G \rightarrow H$ be a homomorphism. We say $f$ is \emph{uniform} if every edge in $H$ has the same number of preimage edges under $f$. The \textbf{uniformization ratio} is defined as:
\[
u(G,H) = \frac{\min_{e' \in E(H)} |f^{-1}(e')|}{\max_{e' \in E(H)} |f^{-1}(e')|}
\]
A perfectly uniform homomorphism has $u(G,H)=1$. We can interpret the concentration parameter $\gamma(G,H)$ in terms of this uniformity:
\[
\gamma(G,H) = \max_{e' \in E(H)} |f^{-1}(e')|
\]
For vertex-transitive graphs $G$, any homomorphism $f : G \rightarrow core(G)$ satisfies uniform vertex fiber sizes:
\[
|f^{-1}(v)| = \frac{|V(G)|}{|V(core(G))|}, \quad \forall v \in V(core(G))
\]
This suggests a natural uniform structure in such graphs and motivates the definition of uniform homomorphisms.

\subsection{Uniformization Perspective}
Uniformization generalizes the behavior of homomorphisms via distributions. Let $\sigma$ be a probabilistic homomorphism. Define $M^\sigma$ as the expected maximum congestion on an edge in $H$. This can help redefine $\gamma(G,H)$ probabilistically:
$$\gamma(G,H) = \inf_{\sigma \in \text{Hom}(G,H)} M^\sigma$$




\section{Extras}
The function $m(G,H) = \max(|G,H|,|H,G|)$ is a metric-like function. It satisfies:
\begin{itemize}
  \item $m(G,H) \geq 0$ (non-negativity)
  \item $m(G,G) = 0$ (partial identity)
  \item $m(G,H) = m(H,G)$ (symmetry)
  \item $m(G,H) + m(H,K) > m(G,K)$ (triangle inequality)
\end{itemize}

To fully satisfy identity, define equivalence classes under mutual homomorphism.




\subsection{Circular Chromatic Numbers}
Inspired by \cite{daneshgar}, the circular chromatic number $\chi_c(G)$ refines $\chi(G)$ and leads to more precise non-homomorphism bounds. Define $G \not\rightarrow H$ if $\chi_c(G) > \chi_c(H)$. Then,
\begin{itemize}
  \item $|G,H| > 0$ whenever $\chi_c(G) > \chi_c(H)$
  \item $\gamma(G,H)$ lower-bounded by circular flow congestion
\end{itemize}


\begin{thebibliography}{9}
\bibitem{kh1} Kaveh Khoshkha (2005), \emph{Nonhomomorphism Factors}, Thesis at Sharif University.
\bibitem{daneshgar} Amir Daneshgar, Hossein Hajiabolhassan (2007), \emph{Circular colouring and algebraic no-homomorphism theorems}, European Journal of Combinatorics, 28(6), 1843–1853.
\bibitem{hn} Pavol Hell, Jaroslav Nešetřil (2004), \emph{Graphs and homomorphisms}, Oxford Lecture Series in Mathematics and Its Applications.
\bibitem{rw} Amir Daneshgar, Hossein Hajiabolhassan (2003), \emph{Graph Homomorphisms Through Random Walks}, JGT 44 (2003) 15–38.
\bibitem{shitov} Yaroslav Shitov (2019), \emph{Counterexamples to Hedetniemi’s conjecture}, Annals of Mathematics, 190(2), 663–667.
\end{thebibliography}

\end{document}
